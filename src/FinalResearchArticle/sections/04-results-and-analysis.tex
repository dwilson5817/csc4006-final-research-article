\subsection{Counting lines of code}
\label{subsec:counting-lines-of-code}

CLOC (count lines of code) is a Perl application which counts the total lines of code across all files in a directory.
It includes breakdowns of which programming language was used.

GitSlice can be used to generate time-series data to understand how the total lines of code in an application changes over time.
One trial example of this would be to use CLOC to show how the total lines of code in its own repository change over time.
To collect this data, GitSlice was run on the Kelvin 2 cluster at Queen's University Belfast.
Analysis was run across $ 5 $ nodes on the cluster with $ 5 $ threads per node.
To run CLOC, the application was built into a Docker image and published to a GitLab container registry.
The application was based on the Perl 5 image, and a copy of CLOC was downloaded into the container.

After analysis, a Python script was used to iterate through each output file and identify the total lines of code at that commit.
This data was saved to a CSV file.
The results are shown in \autoref{fig:cloc}.

\begin{center}
    \begin{tikzpicture}
        \begin{axis}[
            title={Lines of Code in CLOC Application},
            xlabel={Year},
            ylabel={Total Lines of Code},
            date coordinates in=x,
            xticklabel={\year},
            x tick label style={align=center},
        ]
        \addplot table [
            mark=none, col sep=comma,
            trim cells=true,
            y=LOC
        ] {plots/00-cloc.csv};
        \end{axis}
    \end{tikzpicture}
    \label{fig:cloc}
\end{center}

The data shows how the total lines of code grew rapidly during the early development of the program when most of the code was written.
In late 2022 there is a noticeable drop in the total lines of code, likely due to refactoring.
This analysis may be used to developers to identify significant changes to the size of the repository.
For example, if a developer was investigating how to reduce the size of the bundled application, it might be useful to identify the commits at which the size increased significantly which would provide a basis for further investigation as to what was added.

\subsection{Unit tests}
\label{subsec:unit-tests}

Units tests are now an established practice in software engineering.
Developers write unit tests to confirm the correct functionality of a single piece of code (for example, a single method).
It is almost universally expected that any piece of code will have accompanying units tests to confirm it's correct operation.
As such, it might be useful for developers to understand how the total amount of code covered by tests changes over time.
GitSlice run on the Kelvin 2 cluster to execute the unit tests of the last $ 300 $ commits for the Flask project to find code coverage information.
Flask is a web application framework written in Python, which is commonly used for developing websites and HTTP APIs in Python.
Code coverage refers to the total lines in the application which are covered by at least one unit test.

The results are shown in \autoref{fig:unit-tests-numpy}.
Data was again the data collection script outlined in \autoref{subsec:counting-lines-of-code}.
The results show a consistently high rate of code coverage, typically above $ 93 $\%.

\begin{figure}
    \centering
    \begin{tikzpicture}
        \begin{axis}[
            title={Code coverage (\%) in Flask},
            xlabel={Year},
            ylabel={Code Coverage (\%)},
            date coordinates in=x,
            xticklabel={\year},
            x tick label style={align=center},
        ]
        \addplot table [
            mark=none,
            col sep=comma,
            trim cells=true,
            y=Coverage
        ] {plots/01-cov-flask.csv};
        \end{axis}
    \end{tikzpicture}
    \caption{Graph showing how the code coverage changes over time on the Flask project.  Analysis was performed on the last $300$ commits on the main branch.}
    \label{fig:unit-tests-numpy}
\end{figure}



\subsection{Static code analysis}
\label{subsec:static-code-analysis}

Semgrep is a static code analysis tools which aims to find vulnerabilities in dependencies, bugs and enforce code standards.
GitSlice can be used to run Semgrep over many version of a project.
GitSlice was run on the Kelvin 2 cluster on the NumPy project.
NumPy is a Python library which adds support for large multidimensional matrices and along with functions to work with them.
It is a very commonly used library in scientific computing such as physics and mathematics research.

GitSlice was used to perform static code analysis using Semgrep on NumPy.
The commit limit was set to $ 15,000 $ and the commit skip was $ 250 $ meaning every 250th commit of the latest $ 15,000 $ on the main branch was analysed.
The ``auto'' configuration was used.
The results are shown in \autoref{fig:semgrep-numpy}.
Data was again the data collection script outlined in \autoref{subsec:counting-lines-of-code}.

\begin{figure}
    \centering
    \begin{tikzpicture}
        \begin{axis}[
            title={Total Semgrep findings on NumPy},
            xlabel={Year},
            ylabel={Findings},
            date coordinates in=x,
            xticklabel={\year},
            x tick label style={align=center},
        ]
            \addplot table [
                mark=none,
                col sep=comma,
                trim cells=true,
                y=Findings
            ] {plots/02-sast-numpy.csv};
        \end{axis}
    \end{tikzpicture}
    \caption{Graph showing how the total findings from Semgrep change over time on the NumPy project.  Analysis was performed on every $250$th commit for the last $15,000$ commits on the main branch.}
    \label{fig:semgrep-numpy}
\end{figure}

If Semgrep was run when the commits were initially pushed or merged (for example as part of a CI pipeline) then any dependency vulnerability which was not found at the time of the push would not be known.
Using GitSlice allows for each commit to be compared against the most recent vulnerability database and it likely to find more vulnerability.
This maybe important, especially in such a large project as NumPy which will undoubtedly have many previous versions still in use in projects which have not updated their dependencies.
