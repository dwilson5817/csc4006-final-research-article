GitSlice been developed in Python 3.10~\cite{gitlab}.
GitSlice performs analysis inside a Singularity container.
Singularity is a container platform which supports Docker containers.
Therefore, GitSlice supports any Docker container.
The files as they existed at a specific commit are mounted on the container and a command (specified in the configuration file which will be discussed shortly) is run inside the container.
This allows for analysis applications to be bundled as a container and used without the need for its dependencies to be available in the HPC environment.
The output from the container is saved as a text file in the output directory.

\subsection{Configuration}
\label{subsec:configuration}

When run, GitSlices accepts two arguments; the location of a configuration file and the level of logging to use (where logging output is one of the following: debug, info, warning, error and critical).
The configuration file specifies the following settings:

\begin{itemize}
    \item \textbf{Temp Directory}: the temporary directory to use for creating a temporary copy of the repository.
          It is recommended this location is stored locally as opposed to a network location for performance.
          It is not necessary for files in the directory to be able to be accessed by all nodes, but all nodes must be able to create directories at this location.
    \item \textbf{Output Directory}: the directory in which the output from each commit will be saved as a text file to.
    \item \textbf{Output Format}: the format of the output file, for example each file can be named by the time of the commit and the commit ID with an underscore to separate them.
          The time of the commit here refers to the commit time as opposed to author time.
          Output files is saved as a .txt file.
    \item \textbf{Git Repository Type}: indicates if the repository is local or remote, remote repositories will be cloned into the temporary directory and local repositories will be copied into it.
    \item \textbf{Git Repository Source}: the source of the repository, an absolute or relative path for local repositories and a full URI for remote repositories.
    \item \textbf{Analysis Image}: the image the analysis container will be based on.
          This should be a Docker image, which will be pulled and a singularity container will be created from it.
    \item \textbf{Analysis Command}: the command which will be run to start analysis.
    \item \textbf{Repo Directory Name}: the name of the directory which will contain the repository inside the temporary directory.
    \item \textbf{Mount Directory Name}: the name of the directory the virtual filesystem will be mounted at.
    \item \textbf{Commit Limit}: the total amount of commits to go back before stopping - for example a value of $ 15,000 $ means not commits earlier than $ n - 15,000 $ will be analysed (where $ n $ is the total commits from the top of the branch).
    \item \textbf{Commit Skip}: how many commits to skip between analysis, a value of $ 5 $ would mean that every 5th commit will be selected for analysis and a value of $ 1 $ means every commit will be analysed.
\end{itemize}

\subsection{MPI}
\label{subsec:mpi}

GitSlice uses MPI to allow for multiple nodes to work together to analyse the repository.
One instance of GitSlice will be the ``primary'' instance, with $ 0 $ or more ``secondary'' instances which await a commit ID to perform analysis on.
Upon initialisation of GitSlice, the primary instance will analyse the repository to identify the commits to be analysed.
Each instance of GitSlice creates its own directory inside the output directory identified by a UUID generated on initialization.
This enables any amount of instances of GitSlice to work on analysing the repository without overwriting work from another instance.
The work is evenly divided among all instance of GitSlice---for example, if there are $ 5 $ instances with $ 25 $ commits to be analysed then each instance will analysed $ 5 $ commits.
Multiple threads are also used to allow for each instance to analyse multiple commits simultaneously.
For example, if $ 5 $ instances each with $ 5 $ threads then 25 commits will be analysed simultaneously.

\subsection{Filesystem}
\label{subsec:filesystem}

GitSlice uses RepoFS~\cite{repofs} to bring up a virtual filesystem to access the repository.
Each instance mounts it's own virtual filesystem as HPC clusters typically use a network filesystem to keep a consistent filesystem across nodes.
Mounts are not shared across the network so each instance must mount it's own virtual filesystem.
When an instance is finished it's analysis, it will bring the mount down and delete it's directory from the temporary directory.
The output from each analysis will be stored in the output directory.
