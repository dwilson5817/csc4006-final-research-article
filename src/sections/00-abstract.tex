Professional software development teams now almost universally use Git (a distributed version control system) to manage changes to their code.
Git repositories contain metadata relating to the development of the project and show how a project has changed over time.
Analysis of this metadata to gain insight into the software development process is performed in the field of mining software repositories (MSR).
Static code analysis (SCA) tools perform analysis on source code to detect issues during development (they are static because they analyse the code without running it) such as untested or vulnerable code.
SCA only analyses one version of the code, however.
The ability to run code analysis tools across many versions of a project could therefore provide developers with useful information of how their code changes over time.
To achieve this, a tool, GitSlice will be built to iterate through a given set of Git commits to run a code analysis tool on each version of the software and provide the developer with information about how metrics of their code change over time.
Iterating through commits in a linear fashion is slow, so to reduce the time taken to iterate through the commits we will parallelize this process.
